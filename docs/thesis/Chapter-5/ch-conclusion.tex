\chapter{Conclusions and Future Work}

A lot of choice is available for mesh-movement techniques, but only a few will be good for a given application. We conclude that interpolation methods are generally well-suited for unsteady simulations that require mesh movement, as they are very fast. When deformations are relatively small and not highly rotational, Delaunay graph mapping (DGM) is a good choice; when larger and more general deformations are needed, the pure radial basis function (RBF) and to some extent the DGM with RBF interpolation (DGRBF) methods are good. The pure RBF method, while usually giving good robust results, is more expensive than the other interpolation methods considered here. The DGRBF methods, while being inexpensive, are only robust when implemented with angle interpolation, which may be difficult to do for general mesh movements. As for the linear elasticity methods in the context of general mesh movement, they perform reasonably well, and can be better than some interpolation methods such as DGM, but are usually much more expensive than interpolation methods.

RBF, and to some extent stiffened linear elasticity methods, are found to be effective for curved mesh generation, where computational cost is less of an issue. We have used RBF for curved mesh generation with good results for certain turbulent (RANS) flow cases. We find that RBF method is much more cost-effective than Jacobian-stiffened linear elasticity method for comparable results. Of note is that both methods provide `knobs' to tune, such as the basis function and support radius in case of RBF and the stiffening criterion and stiffening exponent for the linear elasticity method. We also conclude that Delaunay graph mapping methods are pretty much unusable for curved mesh generation.

Even though the RBF linear system sometimes cannot be solved by iterative solvers like conjugate gradient and BiCGSTAB, sparse direct solvers have always worked in our experience. Since the number of boundary points is usually smaller by several orders of magnitude than the total number of points, this should usually not be an issue. Good, parallel sparse direct solvers are available, such as SuperLU \cite{superlu}, MUMPS \cite{MUMPS}, PaStiX \cite{pastix} and many more. However, for very large meshes with boundary nodes in the millions, it would be worthwhile to investigate iterative solvers for RBF.

One direction for future work is to complete a surface reconstruction procedure in 3D. For some meshes, the requirement of global $C^2$ continuity is too restrictive. Also, a procedure similar to that described in section \ref{subsec:spline2d} would be quite expensive for surfaces in $\mathbb{R}^3$. We therefore consider local fittings of 2D Taylor polynomials at every boundary vertex, described in \cite{sr:jiaowang} as ``Weighted Averaging of Local Fittings" (WALF). In their paper, Jiao and Wang fit local 2D Taylor polynomials to each vertex of the surface mesh. The coefficients, that is, the derivatives of a local height function in a local coordinate system centered at the vertex, are solved for using vertex position data from a neighborhood of that vertex. It is ensured that there are more neighboring points being considered for data than the number of unknowns to solve for, thereby obtaining an over-determined system of equations. This is solved by a weighted least-squares approach. This is claimed to work well for both smooth surfaces and surfaces with $C^1$ discontinuities (ridges, corners etc.). For the latter case, additional preprocessing of the linear surface mesh is required to detect discontinuities \cite{sr:discontinuities}.
