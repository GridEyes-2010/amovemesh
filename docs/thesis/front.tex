%% ------------------------------ Abstract ---------------------------------- %%
\begin{abstract}

This report describes methods of mesh movement for computational fluid dynamics (CFD) problems. The methods described can be classified into two families - elasticity-based methods and interpolation methods. The former class includes spring analogy methods, and elasticity methods. Interpolation methods described here include the `Delaunay graph' mapping, interpolation by radial basis functions (RBF), and a combination of Delaunay graph mapping and radial basis function interpolation. The merits and disadvantages of these methods have been discussed, and some two-dimensional test cases have been presented. Further, a method for generating curved meshes from linear meshes is presented in this work. The method is designed to generate a curved mesh for any type of grid in a computationally inexpensive and numerically robust manner. In this method, high-order nodes are first placed in the linear mesh, the high-order boundary nodes are then relocated to the true boundary, and finally, the high-order interior nodes are moved using one of the mesh-movement methods, based on the motion of the high-order boundary nodes. The developed method is used to generate curved meshes for a number of geometries. The numerical experiments indicate that the RBF-interpolation method is much better than the prevalent linear-elasticity methods for generating high-order curved meshes judging by curved-mesh quality, computing cost, and ease of implementation.
\end{abstract}


%% ---------------------------- Copyright page ------------------------------ %%
%% Comment the next line if you don't want the copyright page included.
\makecopyrightpage

%% -------------------------------- Title page ------------------------------ %%
\maketitlepage

%% -------------------------------- Dedication ------------------------------ %%
\begin{dedication}
 \centering To my parents.
\end{dedication}

%% -------------------------------- Biography ------------------------------- %%
\begin{biography}
The author was born in Kasauli, HP, India and lived throughout his childhood in Delhi, India. He completed his Bachelor degree in Mechancial Engineering and integrated Masters degree in Biological sciences from Birla Institute of Technology and Science, Pilani, India. During this time, he completed internships related to computational engineering in Indian Institute of Science, Bangalore, and Centre for Fuel Cell Technology, International Advanced Research Centre for Powder Metallurgy and New Materials, Chennai, India. In August 2014, he began the MS Mechanical Engineering program at NC State University, and went on to start work at the Computational Fluid Dynamics Laboratory under the guidance of Professor Hong Luo.
\end{biography}

%% ----------------------------- Acknowledgements --------------------------- %%
\begin{acknowledgements}
First and foremost, I would like express my heartfelt thanks to my advisor Professor Hong Luo for guiding and advising me on various issues throughout my time at NC State. Next, I thank my committee members Professor Jack Edwards Jr and Professor Pierre Gremaud, for teaching me important concepts and agreeing to be on my MS defense committee. I am also thankful to my colleagues at the CFD lab - Aditya Pandare, Xiaodong Liu, Chuanjin Wang, Chad Rollins, Jialin Lou, Dr Xiaoquan Yang and Dr Jian Cheng for always entertaining my questions and discussing things when required. I am especially thankful to Xiaodong for running RDGFLO on some of my meshes, many interesting discussions and also the rides home. Finally, I am very thankful to my parents for being very supportive and for encouraging me to do my best.
\end{acknowledgements}


\thesistableofcontents

\thesislistoftables

\thesislistoffigures
